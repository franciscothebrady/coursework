% Options for packages loaded elsewhere
\PassOptionsToPackage{unicode}{hyperref}
\PassOptionsToPackage{hyphens}{url}
\documentclass[
]{article}
\usepackage{xcolor}
\usepackage{stata}
\usepackage{amsmath,amssymb}
\setcounter{secnumdepth}{-\maxdimen} % remove section numbering
\usepackage{iftex}
\ifPDFTeX
  \usepackage[T1]{fontenc}
  \usepackage[utf8]{inputenc}
  \usepackage{textcomp} % provide euro and other symbols
\else % if luatex or xetex
  \usepackage{unicode-math} % this also loads fontspec
  \defaultfontfeatures{Scale=MatchLowercase}
  \defaultfontfeatures[\rmfamily]{Ligatures=TeX,Scale=1}
\fi
\usepackage{lmodern}
\ifPDFTeX\else
  % xetex/luatex font selection
\fi
% Use upquote if available, for straight quotes in verbatim environments
\IfFileExists{upquote.sty}{\usepackage{upquote}}{}
\IfFileExists{microtype.sty}{% use microtype if available
  \usepackage[]{microtype}
  \UseMicrotypeSet[protrusion]{basicmath} % disable protrusion for tt fonts
}{}
\makeatletter
\@ifundefined{KOMAClassName}{% if non-KOMA class
  \IfFileExists{parskip.sty}{%
    \usepackage{parskip}
  }{% else
    \setlength{\parindent}{0pt}
    \setlength{\parskip}{6pt plus 2pt minus 1pt}}
}{% if KOMA class
  \KOMAoptions{parskip=half}}
\makeatother
\setlength{\emergencystretch}{3em} % prevent overfull lines
\providecommand{\tightlist}{%
  \setlength{\itemsep}{0pt}\setlength{\parskip}{0pt}}
\usepackage{bookmark}
\IfFileExists{xurl.sty}{\usepackage{xurl}}{} % add URL line breaks if available
\urlstyle{same}
\hypersetup{
  pdftitle={Problem Set 1},
  pdfauthor={Francisco Brady},
  hidelinks,
  pdfcreator={LaTeX via pandoc}}

\title{Problem Set 1}
\author{Francisco Brady}
\date{25 Jan 2025}

\begin{document}
\maketitle

\begin{stlog}
. *set graphics off
. use solis_dataset.dta
{\smallskip}
. 
\end{stlog}

\subsubsection{1. Generate three variables for time period
1:}\label{generate-three-variables-for-time-period-1}

\paragraph{a. An indicator that flags students scoring above the cutoff
of
475}\label{a.-an-indicator-that-flags-students-scoring-above-the-cutoff-of-475}

\begin{stlog}
. 
\end{stlog}

\paragraph{b. An indicator for ``pre-selected'' status. Pre-selected
students are those for whom income quintile is less than
5}\label{b.-an-indicator-for-pre-selected-status.-pre-selected-students-are-those-for-whom-income-quintile-is-less-than-5}

\paragraph{c.~A running or forcing variable centered at the cutoff score
of
475}\label{c.-a-running-or-forcing-variable-centered-at-the-cutoff-score-of-475}

\subsubsection{2. Calculate some descriptive statistics and describe
what you
find:}\label{calculate-some-descriptive-statistics-and-describe-what-you-find}

\paragraph{a. For time period 1, how many individuals are
``pre-selected?'' What proportion of PSU takers are pre-selected? What
proportion of those scoring below/above the cutoff are
pre-selected?}\label{a.-for-time-period-1-how-many-individuals-are-pre-selected-what-proportion-of-psu-takers-are-pre-selected-what-proportion-of-those-scoring-belowabove-the-cutoff-are-pre-selected}

\paragraph{b. Summarize and plot the distribution of the forcing
variable (or of the PSU score) for time period 1. Briefly describe what
you find. Do you see any evidence of bunching or manipulation around the
475-point
threshold?}\label{b.-summarize-and-plot-the-distribution-of-the-forcing-variable-or-of-the-psu-score-for-time-period-1.-briefly-describe-what-you-find.-do-you-see-any-evidence-of-bunching-or-manipulation-around-the-475-point-threshold}

\paragraph{c.~Based on time period 1, calculate the rates of immediate
enrollment (enrolt1) and ever enrollment (everenroll1) for 3 groups:
non-pre-selected students and, among pre-selected students, those above
and below the 475-point PSU cutoff. Do this only for observations that
have a non-missing value for PSU in time period
1.}\label{c.-based-on-time-period-1-calculate-the-rates-of-immediate-enrollment-enrolt1-and-ever-enrollment-everenroll1-for-3-groups-non-pre-selected-students-and-among-pre-selected-students-those-above-and-below-the-475-point-psu-cutoff.-do-this-only-for-observations-that-have-a-non-missing-value-for-psu-in-time-period-1.}

\paragraph{d.~Calculate the rate of immediate and ever enrollment for
all students by family income quintile (again, among those with a value
for PSU in time period
1).}\label{d.-calculate-the-rate-of-immediate-and-ever-enrollment-for-all-students-by-family-income-quintile-again-among-those-with-a-value-for-psu-in-time-period-1.}

\subsubsection{3. There are two key assumptions to a regression
discontinuity analysis: (1) the likelihood of being assigned to the
treatment varies discontinuously through the cutoff; and (2)
characteristics that are associated with the outcome of interest change
smoothly through the cutoff. Present evidence (figures and/or tables) of
assumption (2) by analyzing the distribution of scores across family
income quintiles. Briefly discuss your findings. See the bottom panel of
Figure A1 in the paper for an example. Also discuss any remaining
sources of bias that your RD analysis cannot rule
out.}\label{there-are-two-key-assumptions-to-a-regression-discontinuity-analysis-1-the-likelihood-of-being-assigned-to-the-treatment-varies-discontinuously-through-the-cutoff-and-2-characteristics-that-are-associated-with-the-outcome-of-interest-change-smoothly-through-the-cutoff.-present-evidence-figures-andor-tables-of-assumption-2-by-analyzing-the-distribution-of-scores-across-family-income-quintiles.-briefly-discuss-your-findings.-see-the-bottom-panel-of-figure-a1-in-the-paper-for-an-example.-also-discuss-any-remaining-sources-of-bias-that-your-rd-analysis-cannot-rule-out.}

\subsubsection{4. Replicate columns 1 and 2 of Table 3, where the
outcome is immediate college enrollment. Put these findings in a nice,
clear table with all necessary information including bandwidth used.
Briefly explain the relevant coefficient(s) in each
column.}\label{replicate-columns-1-and-2-of-table-3-where-the-outcome-is-immediate-college-enrollment.-put-these-findings-in-a-nice-clear-table-with-all-necessary-information-including-bandwidth-used.-briefly-explain-the-relevant-coefficients-in-each-column.}

\subsubsection{\texorpdfstring{5. The loan eligibility rule lends itself
to a ``sharp'' RD specification in the short term. However, the fact
that individuals may retake the test and become eligible in later years
introduces some ``fuzziness'' to the treatment assignment. Use a 2SLS RD
setup where exceeding the threshold in year 1 is an instrument for
\texttt{everelig1} to replicate Table 4 columns 1 and 2. Report
first-stage results as well. What do you infer from column 1-2 results?
Is this consistent with your estimates from question
4?}{5. The loan eligibility rule lends itself to a ``sharp'' RD specification in the short term. However, the fact that individuals may retake the test and become eligible in later years introduces some ``fuzziness'' to the treatment assignment. Use a 2SLS RD setup where exceeding the threshold in year 1 is an instrument for everelig1 to replicate Table 4 columns 1 and 2. Report first-stage results as well. What do you infer from column 1-2 results? Is this consistent with your estimates from question 4?}}\label{the-loan-eligibility-rule-lends-itself-to-a-sharp-rd-specification-in-the-short-term.-however-the-fact-that-individuals-may-retake-the-test-and-become-eligible-in-later-years-introduces-some-fuzziness-to-the-treatment-assignment.-use-a-2sls-rd-setup-where-exceeding-the-threshold-in-year-1-is-an-instrument-for-everelig1-to-replicate-table-4-columns-1-and-2.-report-first-stage-results-as-well.-what-do-you-infer-from-column-1-2-results-is-this-consistent-with-your-estimates-from-question-4}

\subsubsection{6. One of the nice features of this paper is that the RDD
findings so clearly show the main result. Create your own version of
Figure 1. {[}Just to warn you, you almost certainly will NEVER get an RD
graph that looks this
clean!{]}}\label{one-of-the-nice-features-of-this-paper-is-that-the-rdd-findings-so-clearly-show-the-main-result.-create-your-own-version-of-figure-1.-just-to-warn-you-you-almost-certainly-will-never-get-an-rd-graph-that-looks-this-clean}

\subsubsection{7. This final question asks you to estimate a series of
placebo effects to gauge the size of the enrollment discontinuity at a
score of 475 relative to other discontinuities at irrelevant
scores.}\label{this-final-question-asks-you-to-estimate-a-series-of-placebo-effects-to-gauge-the-size-of-the-enrollment-discontinuity-at-a-score-of-475-relative-to-other-discontinuities-at-irrelevant-scores.}

\paragraph{\texorpdfstring{a. First, estimate the Table 3 column 1
specification for every value of \(\tau\) between 431 and 519. That is,
substitute placebo values of \(\tau\) in the equation (1) term
\(1(T_i \geq \tau)\). Use a 44-unit bandwidth as in the main results,
but note that this bandwidth will cover different PSU values in each
placebo estimate. Store coefficients for the \(1(T_i \geq \tau)\)
indicator (\(\beta_1\)), and plot the distribution of placebo
coefficients. Mark where the true effect of \(\tau = 475\) lies in the
distribution of placebo effects. What share of placebo effects are
smaller in absolute value than the true
\(\beta_1\)?}{a. First, estimate the Table 3 column 1 specification for every value of \textbackslash tau between 431 and 519. That is, substitute placebo values of \textbackslash tau in the equation (1) term 1(T\_i \textbackslash geq \textbackslash tau). Use a 44-unit bandwidth as in the main results, but note that this bandwidth will cover different PSU values in each placebo estimate. Store coefficients for the 1(T\_i \textbackslash geq \textbackslash tau) indicator (\textbackslash beta\_1), and plot the distribution of placebo coefficients. Mark where the true effect of \textbackslash tau = 475 lies in the distribution of placebo effects. What share of placebo effects are smaller in absolute value than the true \textbackslash beta\_1?}}\label{a.-first-estimate-the-table-3-column-1-specification-for-every-value-of-tau-between-431-and-519.-that-is-substitute-placebo-values-of-tau-in-the-equation-1-term-1t_i-geq-tau.-use-a-44-unit-bandwidth-as-in-the-main-results-but-note-that-this-bandwidth-will-cover-different-psu-values-in-each-placebo-estimate.-store-coefficients-for-the-1t_i-geq-tau-indicator-beta_1-and-plot-the-distribution-of-placebo-coefficients.-mark-where-the-true-effect-of-tau-475-lies-in-the-distribution-of-placebo-effects.-what-share-of-placebo-effects-are-smaller-in-absolute-value-than-the-true-beta_1}

\paragraph{b. Repeat part 7a, but for the Table 3 column 2
specification.}\label{b.-repeat-part-7a-but-for-the-table-3-column-2-specification.}

\begin{stlog}
. 
\end{stlog}

\end{document}
